%%%%%%%%%%%%%%%%%%%%%%%%%%%%%%%%%%%%%%%%%%%%%%%%%%%%%%%%%%%%%%%%%%%%%
%%%%%%%%%%%%%%%%%%%%%%%%%%%%%%%%%%%%%%%%%%%%%%%%%%%%%%%%%%%%%%%%%%%%%
%%%                                                                                                                                                                                                           %%%
%%%   NOTE!!!!!!!!!!!!!!! If you get some compile error that refers to a runaway argument, just remove writeup.aux and compile again!  %%%
%%%                                                                                                                                                                                                            %%%
%%%%%%%%%%%%%%%%%%%%%%%%%%%%%%%%%%%%%%%%%%%%%%%%%%%%%%%%%%%%%%%%%%%%%%
%%%%%%%%%%%%%%%%%%%%%%%%%%%%%%%%%%%%%%%%%%%%%%%%%%%%%%%%%%%%%%%%%%%%%%

%%%%%%%%%%%%%%
%% Run LaTeX on this file several times to get Table of Contents,
%% cross-references, and citations.

%%%%%%%%%%%%%
% 7x10
\documentclass{wileySev}

\usepackage{graphicx}

\usepackage{hyperref}
\hypersetup{
    colorlinks=true,       % false: boxed links; true: colored links
    linkcolor=blue,          % color of internal links (change box color with linkbordercolor)
    citecolor=green,        % color of links to bibliography
    filecolor=magenta,      % color of file links
    urlcolor=red           % color of external links
}

% For PostScript text
\usepackage{w-bookps}

\setcounter{secnumdepth}{3}

\setcounter{tocdepth}{2}

\newcommand{\VT}[1]{\ensuremath{{V_{T#1}}}}

\newbox\sectsavebox
\setbox\sectsavebox=\hbox{\boldmath\VT{xyz}}

\begin{document}

\offprintinfo{Probabilistic Accent Detection Using HMMs}{Calhoun, Parker, Vaslas, and Vera}

\booktitle{Spoken Language Accent Detection}
\subtitle{Probabilistic Accent Detection Using Hidden Markov Models}

\authors{Tre Calhoun\\ %Sorry, Tre! I don't know how to use accented letters in LaTeX
La Vesha Parker\\
Andrew Vaslas\\
Nicolas Vera\\
\affil{Cornell University}
}

\titlepage
\tableofcontents


\begin{preface}
All information presented within this document represents our exploration of the HTK software. We make no guarantee of things things things

\end{preface}

\begin{introduction}
% Discuss our motivation, etc.
This is the introduction.
This is the introduction.
This is the introduction.
This is the introduction.
This is the introduction.
This is the introduction.


% Any sources you mention from the motivation
\begin{chapreferences}{3.}
\bibitem{zkilby}J. S. Kilby,
``Invention of the Integrated Circuit,'' {\it IEEE Trans. Electron Devices,}
{\bf ED-23,} 648 (1976).

\bibitem{zhamming}R. W. Hamming,
                 {\it Numerical Methods for Scientists and 
                 Engineers}, Chapter N-1, McGraw-Hill, 
                 New York, 1962.

\bibitem{zHu}J. Lee, K. Mayaram, and C. Hu, ``A Theoretical
               Study of Gate/Drain Offset in LDD MOSFETs''
                     {\it IEEE Electron Device Lett.,} {\bf EDL-7}(3). 152 
                     (1986).
\end{chapreferences}
\end{introduction}


\part[HTK Software Suite]
{Hidden Markov Model\\Toolkit Software Suite}

\chapter{Installation of HTK Software}
talk about installation stuff here, I guess
\section{Mac OSX}
\section{Windows}

\chapter{Training Corpus Acquisition}

\chapter{Training Corpus with HTK}

\prologue{The sheer volumne of answers can often stifle insight...The purpose
of computing\index{computing!the purpose} is insight, not numbers.}
{Hamming \cite{hamming}}


\section{Record or Input Sound Files}
Here is some text.


\section{Labeling the Sound Files}
Here are some things you can do for a special
section head.

\section{General Remarks}
Here is some normal text.
Here is some normal text.
Here is some normal text.



\chapter{Coding the Data}

\section{Mel Frequency Cepstral Coefficients}
Here we describe what a MFCC is and its usefulness to us.

\section{Obtaining .mfcc Files}

\subsection{Configuration File}
Screenshot of the configuration file along with justification of the various parameters

\subsection{Command Line Actions}

\subsubsection{The Creation of targetlist.txt}


\chapter{Setting Parameters for the Hidden Markov Model}
Multiple things should happen here:
\begin{enumerate}
\item Explain what an HMM is and what it is useful for
\item Explain particularly why it works for what we are doing
\item Describe the input parameters to a hidden markov model
\item Explain why we made any changes to what the original tutorial had/any issues we encountered (i.e. errors being raised when we tried to have too many states due to not having enough training examples for all those states)
\end{enumerate}

I have some sample sections below following the list above:

\section{What is a Hidden Markov Model?}
Here is some sample text.

\section{HMMs and Accent Detection}
Lorem ipsum Lorem ipsum Lorem ipsum Lorem ipsum Lorem ipsum Lorem ipsum

\section{Input Parameters to HMMs} 
Lorem ipsum Lorem ipsum Lorem ipsum Lorem ipsum Lorem ipsum Lorem ipsum

\section{Justification for our Modifications}

\section{Summary}
This is a summary of this chapter.
Here are some references: \cite{xkilby}, \cite{xberen}.

\begin{chapreferences}{5.}
\bibitem{xkilby}J. S. Kilby,
``Invention of the Integrated Circuit,'' {\it IEEE Trans. Electron Devices,}
{\bf ED-23,} 648 (1976).


\bibitem{xhamming}R. W. Hamming,
                 {\it Numerical Methods for Scientists and 
                 Engineers}, Chapter N-1, McGraw-Hill, 
                 New York, 1962.

\bibitem{xHu}J. Lee, K. Mayaram, and C. Hu, ``A Theoretical
               Study of Gate/Drain Offset in LDD MOSFETs''
                     {\it IEEE Electron Device Lett.,} {\bf EDL-7}(3). 152 
                     (1986).

\bibitem{xberen}A. Berenbaum, 
B. W. Colbry, D.R. Ditzel, R. D Freeman, and 
K.J. O'Connor, ``A Pipelined 32b Microprocessor with 13 kb of Cache Memory,''
{it Int. Solid State Circuit Conf., Dig. Tech. Pap.,} p. 34 (1987).
\end{chapreferences}

\chapter{Defining the Grammar of Your Network}
\section{What does that even mean}
\section{Define your Grammar}
\section{Define your Dictionary}
\section{Generating the Network}

\chapter{Testing with New Samples}
Corresponds to the Recognition chapter (Moreau ch. 7)


\part{Data Visualization}
% If we decide to go crazy

\part{Error Handling and General Tips}

\appendix{Error Handling}
\markboth{A short guide to understanding and handling errors}{A short guide to understanding and handling errors}
This is an appendix with a title.
\begin{equation}
\alpha\beta\Gamma\Delta
\end{equation}



\begin{figure}[ht]
\caption{This is an appendix figure caption.}
\end{figure}


\begin{table}[ht]
\caption{Appendix table caption}
\centering
\begin{tabular}{cccc}
\hline
Alpha&Beta&Gamma&Delta\\
\hline
$\alpha$&$\beta$&$\Gamma$&$\Delta$\\
\hline
\end{tabular}
\end{table}

\appendix{Software Used}
Just list all software used and why we used it
\begin{enumerate}
\item Audacity
\item HTK --> Maybe even list each of the things we used under HTK \& why, i.e. HLab for labeling, HParse for whatever
\item was there anything else?
\end{enumerate}


\appendix{References}
% Just contains a sample reference for HMMs on Wikipedia. 
\begin{references}{3.}
\bibitem{link_used_elsewhere}Random People,
\href{http://en.wikipedia.org/wiki/Hidden_Markov_model}{``Hidden Markov Model,''}(2014).

\bibitem{hamming}R. W. Hamming,
                 {\it Numerical Methods for Scientists and 
                 Engineers}, Chapter N-1, McGraw-Hill, 
                 New York, 1962.

\bibitem{Hu}J. Lee, K. Mayaram, and C. Hu, ``A Theoretical
               Study of Gate/Drain Offset in LDD MOSFETs''
                     {\it IEEE Electron Device Lett.,} {\bf EDL-7}(3). 152 
                     (1986).

\bibitem{beren}A. Berenbaum, 
B. W. Colbry, D.R. Ditzel, R. D Freeman, and 
K.J. O'Connor, ``A Pipelined 32b Microprocessor with 13 kb of Cache Memory,''
{it Int. Solid State Circuit Conf., Dig. Tech. Pap.,} p. 34 (1987).
\end{references}


\printindex


\end{document}

